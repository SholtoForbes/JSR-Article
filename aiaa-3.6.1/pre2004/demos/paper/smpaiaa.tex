%
% latex2e adocument: smpaiaa.tex
%
%  sample AIAA conference paper, journal article,
%  journal note, and journal submission
%
%  -- bil kleb <10 jan 97>
%

\documentclass[cover]{aiaa}% options: paper, article, note,
                           %          cover, and/or submit

% note: until we reach \begin{document}, we're in the `preamble'

% load items used by the aiaa class:

% for a conference paper (`paper' option [default]):
\PaperNumber{98--0879}
\PaperNotice{\CopyrightD{1999}}
\CoverFigure{smpfig}% [optional cover figure]
\Conference{{\bfseries 36th AIAA Aerospace Sciences \\
            Meeting and Exhibit} \\
            January~12--15,~1998/Reno,~NV}

% for journal submission (`submit' option):
\SubmitName{Kleb ET AL.\ (A10659)}

% for journal simulation (`article' or `note' options):
\JournalName{Journal of Spacecraft and Rockets}
\JournalPage{39}
\JournalIssue{Volume~36, Number~2}
\JournalNotice{Presented as Paper~98--0879 at the AIAA
               36th Aerospace Sciences Meeting, Reno,~NV,
               Jan.~12--15,~1998; received Feb.~13,~1998;
               revision received Sept.~18,~1998;
               accepted for publication Oct.~1,~1998.
               \CopyrightD{1999}}

% specific to journal article:
\ArticleIssue{Vol.~36, No.~2, March--April 1999}% first page
\ArticleHeader{Kleb ET AL.}%                      subsequent pages

% specific to journal note:
\NoteHeader{J.Spacecraft, Vol.~36, No.~2: Engineering Notes}

% load the title, author, and abstract for use with the \maketitle command

\title{Simulation of an Aerospace Vehicle \\
       Pitch-Over Maneuver}
                                
\author{
%
William L.~Kleb%
%
  \thanks{Research Engineer, Aerothermodynamics Branch,
          Aero- and Gas-Dynamics Division,
          Research and Technology Group.} \\
%
         {\itshape NASA Langley Research Center,
                   Hampton,~VA~23681}\\[5pt]
%
A.~N.~Author%
%
  \thanks{Deligent worker,
          AIAA member.}
%
\ and Y.~F.~Anotherlongername%
%
  \thanksibid{2} \\ % use same footnote as second author
%
         {\itshape Someother Affliation,
         Atown,~ST~98293}
}

\abstract{{\em Note: this abstract does not appear when a
journal note simulation is the chosen option.}  The objective of
the present work is to summarize the application of unsteady
computational fluid dynamic methods to the problem of predicting
verticle take-off/vertical landing vehicle aerodynamics during an
un-powered pitch-over maneuver.  In addition to the
time-dependent simulation of a pitch-over maneuver, a series of
steady solutions at discrete points are also computed for
comparison with wind-tunnel measurements and as a means of
quantifying unsteady effects.  As this application represents a
new challenge to unsteady computational fluid dynamics,
observations concerning grid resolution, far-field boundary
placement, temporal resolution, and the suitability of assuming
flow-field symmetry are discussed.}


\begin{document}

\maketitle%  create the cover page, title block, and notices

\section{Nomenclature}% you could have this generated
                      % automatically via the nomenclature package

\begin{tabbing}
12345678 \= Reynolds number based on length $s$ \kill
$c$      \> Sound speed, m/s \\
$M$      \> Mach number \\
$P$      \> Pressure, Pa \\
$Re_s$   \> Reynolds number based on length $s$ \\
$T$      \> Temperature, K \\
$V$      \> Velocity, m/s \\
$x,y,z$  \> Cartesian body axes, m \\
$\alpha$ \> Angle of attack, deg \\
$\eta$   \> Wall-normal distance, m \\
$\rho$   \> Density, kg/m$^3$
\end{tabbing}

\subsection{Subscripts}
\begin{tabbing}
12345678 \= \kill
$tran$   \> Transition \\
$w$      \> Wall \\
$\infty$ \> Freestream
\end{tabbing}

\subsection{Superscripts}
\begin{tabbing}
12345678  \= \kill
$0$       \> Fiduciary point \\
$n\!+\!1$ \> Time level
\end{tabbing}

\section{Introduction}

\dropword NASA's Access to Space Study\cite{bekey:94ja}
recommends the development of a {\em fully}\ reusable launch
vehicle\cite{dornheim:94awst} to replace the aged Space
Shuttle. {\em The \verb+\dropword+ command created the hanging
capital letter and automatically capitalizes the rest of the
word.}  A method of reaching this goal is to develop a vehicle
which does not rely on expendable boosters to reach orbit, a
single-stage-to-orbit vehicle\cite{austin:94ja}.  One such
configuration being investigated is a Vertical Take-off and
Vertical Landing (VTVL) concept\cite{austin:94ja}.  In one
scenario, the VTVL vehicle, upon completion of its mission in
low-Earth orbit, reenters nose first, decelerates to subsonic
speeds, and then performs a rotation
maneuver\cite{dornheim:95awst,david:95sn,dornheim:95awst2} to
land vertically.


Figure~\ref{f:trn2lndg} presents a schematic of the last portion
of a typical VTVL entry.
\begin{figure}
  \incfig{smpfig}
  \caption{Transition to landing for a VTVL vehicle.}
  \label{f:trn2lndg}
\end{figure}
The pitch-over maneuver, which occurs near Mach 0.2, is
characterized by high angle of attack, unsteady, vortical flow.
Accurately predicting vehicle performance during this aerodynamic
pitch-over maneuver is quite challenging. While ground-based
facilities can readily predict the vehicle's aerodynamics at
discrete points during the maneuver, simulating the transient
motion in a wind tunnel is difficult.\cite{oleary:94cp}
Time-dependent Computational Fluid Dynamics (CFD) offers another
means of analyzing the pitch-over maneuver. The majority of work
in unsteady CFD has, however, been restricted to small amplitude,
harmonic variations in angle of attack in support of aeroelastic
flutter predictions.\cite{edwards:92cp}

The objective of the present work is to summarize the application
of unsteady CFD methods to the problem of predicting VTVL vehicle
aerodynamics during an un-powered pitch-over maneuver (further
details are available in the companion conference
paper\cite{kleb:96cp}).  In addition to the time-dependent
simulation of a pitch-over maneuver, a series of steady solutions
at discrete points are also computed for comparison with
wind-tunnel measurements and as a means of quantifying unsteady
effects.  As this application represents a new challenge to
unsteady CFD, observations concerning grid resolution, far-field
boundary placement, temporal resolution, and the suitability of
assuming flow-field symmetry are documented in
Ref.~\citen{kleb:96cp}.

\section{Geometry}

The vehicle's fore body is an 8 deg half-angle sphere cone with
nose radius equal to 0.3 of the base radius.  The aft body,
beginning at the 85 percent fuselage station, is a cylinder with
a partially squared-off cross-section producing flat ``slices''
extending from the base of the vehicle to approximately the 60
percent fuselage station.  The vehicle has a fineness ratio of
6.4.  A complete description of the vehicle geometry modeled has
been given by Woods in Ref.~\citen{woods:95cp}.

{\em Through the generosity of Karen Bibb, we have a
demonstration of a subfigure situation.  Note that with imbedded
labels you can refer to Fig.~\ref{fig:both} as a whole or
specifically, things in Fig.~\ref{fig:first} or
Fig.~\ref{fig:second}.}  An early Lockheed-Martin X-33
configuration was used in the remainder of the examples.  The
full vehicle is shown in Fig.~\ref{fig:both}.
\begin{figure}
  \begin{subfigmatrix}{1}
    \subfigure[\bf First pretty picture.]
              {\incfig{smpfig}\label{fig:first}}
    \subfigure[\bf Second pretty picture.]
              {\incfig{smpfig}\label{fig:second}}
  \end{subfigmatrix}
  \caption{Temperature distribution comparisons at various wing
           semi-span stations as a function of chord.  Whuh?}
  \label{fig:both}
\end{figure}
This configuration (B1001A) was evaluated during
Phase I of the X-33 program.  It has twin vertical tails, fins,
and outboard body flaps.  The engines are modeled by the box-shaped
structure on the base.  {\em We can also have a ``table'' of two,
or four, or more figures as in Fig.~\ref{fig:four}.}
\begin{figure}
  \begin{subfigmatrix}{2}
    \subfigure[\bf First pretty picture.]
      {\incfig{smpfig}}
    \subfigure[\bf Second pretty picture.]
      {\incfig{smpfig}}
    \subfigure[\bf Second pretty picture, again.]
      {\incfig{smpfig}}
    \subfigure[\bf First pretty picture, again.]
      {\incfig{smpfig}}
 \end{subfigmatrix}
  \caption{Four small figures in a table-like setting.}
  \label{fig:four}
\end{figure}

\section{Computational Mesh}

The underlying surface definition database was generated from
structured surface patches obtained using
GRIDGEN\cite{GRIDGEN3D_release}, GridTool\cite{GridTool}, and
simple analytical methods.  The unstructured surface and
flow-field grids were then generated using
FELISA\cite{peraire:90cp} and TETMESH.\cite{kennon:92cp} The
coarsest mesh has 32,374 tetrahedra with 6,634 nodes.  Additional
meshes are described in Ref.~\citen{kleb:96cp}; however, all the
results shown here are the result of the coarsest meshes.

Since the flow about symmetric configurations at high angles of attack,
even with zero side-slip, often involve asymmetric, vortex-dominated,
features,\cite{yoshinaga:94cp,cobleigh:94cp,dusing:93cp,fisher:94cp}
two different options for the computational domain were employed:
one modeling the complete vehicle and another modeling only half
of the vehicle, assuming symmetry across the pitch-plane.  This
aspect of the study is covered in Ref.~\citen{kleb:96cp}.

\section{Numerical Method}

The 3D3U code of Batina\cite{batina:93aij} was used exclusively in this
study. The 3D3U code was originally developed to study harmonically
pitching wings and wing-bodies in transonic flow.  The code can
incorporate aeroelastic effects through assumed mode shapes, coupled
with a deforming mesh via the linear spring analogy.

The following defaults were used for the computed results in this
study: Roe's flux-difference splitting, an eigenvalue limiter
threshold value of 0.3, second-order flux reconstruction using a
$\kappa$ of 0.5, and Gauss-Seidel implicit time integration with
a CFL number of one million.

\section{Maneuver Definition}

The pitch schedule chosen for this study is the first half of a
sine function.  Initially, the vehicle is in steady flight at
17.5 degrees angle of attack, Mach 0.2.  At time zero, the
vehicle begins the pitch-over maneuver, reaching a maximum pitch
rate exactly half-way through the maneuver and finishing at a 180
deg angle of attack.  For this study, the time to complete the
maneuver was chosen as 90 seconds, giving a maximum rotation rate
of 3.1 deg per second halfway through the maneuver.  For
simplicity, it is assumed that the free-stream Mach number
remains constant throughout the maneuver.

{\em Sticking in a table for guidance (see Table~\ref{tab:sample}).}
\begin{table}[htbp]
  \begin{center}
    \caption{A sample table}
    \begin{tabular}{ccc} \hline\hline
      \multicolumn{2}{c}{Header 1} & Header 2 \\ \hline
      a & b & c \\
      d & e & f \\ \hline\hline
    \end{tabular}
    \label{tab:sample}
  \end{center}
\end{table}
Normally there would be text following the table, so that it
is not left ``hanging'' into the next section.

\section{Results}

The main results are presented in two stages which are followed
by comments on flow asymmetries for both steady and unsteady
flows.  The first stage is computed steady data as it compares to
experimental results.  while the second is a comparison of steady
to unsteady data.

\subsection{Steady Flow}

As a baseline to examine the unsteady effects of the pitch-over
maneuver itself, steady flow at selected angles of attack were
computed.  Figure~\ref{f:expcomp} shows a comparison of the
normal force coefficient with the experimental data of
Woods\cite{woods:95cp} for angles of attack from 0 to 60 deg.
\begin{figure}
   \incfig{smpfig}
   \caption{Comparison of steady normal force coefficient as a
     function of angle of attack with experimental data of
     Woods\protect{\cite{woods:95cp}}}
   \label{f:expcomp}
\end{figure}
The computed results agree well (within 20 percent) at small angles of attack,
and diverge from the experimental results as the angle
of attack increases largely due to the position of the lee-side
separation line---a viscous phenomenon.
Since the computed results are modeling inviscid
flow, the only mechanism for flow separation is the numerical
dissipation in the scheme.  Thus, the exact location of the computed
separation line is highly grid and scheme dependent. Compounding
this is the fact that the lee-side separation line nearly
encompasses the length of the vehicle; and thus, a slight
deviation can make a large difference in the integrated coefficients.

\subsection{Unsteady Flow}

Figure~\ref{f:aerocomp} compares the normal force coefficient of
both the steady and unsteady calculations.
\begin{figure*}
  \incfig{smpfig}
  \caption{Comparison of steady and unsteady normal force
    coefficients as a function of angle of attack.  {\em Example
    of a figure that spans both columns.  The danger is that the
    figure numbering may be out of order since the single-column
    float and double-column float counters are not connected when
    it comes to determining placement order.  You can correct this
    known ``feature'' of \LaTeX{} by using the \texttt{fix2col} package.}}
 \label{f:aerocomp}
\end{figure*}
The solid line represents the unsteady results and the symbols
are the steady-flow results.  Readily discernible is the fact
that the steady and unsteady results are significantly different.
The time-dependent results have the time lag behavior expected
for moderately unsteady flow: showing the same general
qualitative trend throughout the angle-of-attack range, but with
the unsteady results lagging behind the steady results. Other
aerodynamic coefficients show similar
differences.\cite{kleb:96cp}

\subsection{Flow Asymmetry}

As documented in Ref.~\citen{kleb:96cp}, no appreciable
asymmetries were found for the steady flow cases computed
although they were present in the experimental data of
Woods\cite{woods:95cp}.  However, for the unsteady case,
asymmetric flow is apparent as shown in Figure{f:uasym}.
\begin{figure}[t]
  \incfig{smpfig}
  \caption{Unsteady flow asymmetry: side force coefficient as a
    function of angle of attack.}
  \label{f:uasym}
\end{figure}
This figure shows the appearance of a non-zero side-force similar
to that reported for {\em steady} flow by Woods\cite{woods:95cp}.
The most significant manifestations of the asymmetries occur in
the 90 to 135 deg angle-of-attack range.

\section{Concluding Remarks}

The objective of the present work was to focus an unsteady CFD
method on the prediction of VTVL vehicle aerodynamics during a
pitch-over maneuver.  This was accomplished through the use of
the inviscid 3D3U code\cite{batina:93aij} A series of steady
solutions at discrete points in the maneuver were computed and it
was shown that, even for the unrealistically slow pitch-over rate
studied, unsteady effects were large.  More importantly, the
rotation maneuver creates flow asymmetries which lead to side
forces not apparent for the steady cases.

As this is an exploratory study, there is certainly room for
future work. The following is just a handful of extensions which
would be necessary to create an effective design tool:
incorporating viscous effects, allowing movable control surfaces,
coupling a six-degree-of-freedom rigid body dynamics solver,
adding control law, and incorporating an adaptive grid
capability.

\bibliography{smpbtx}
\bibliographystyle{aiaa}

\end{document}
