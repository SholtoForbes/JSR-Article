\documentclass{AIAA}

\begin{document}

\title{Preparation of Papers for AIAA Technical Journals}

\author{First A. Author\footnote{Insert Job Title, Department Name, Address/Mail Stop, and AIAA Member Grade (if any) for first author.} and Second B. Author Jr.\footnote{Insert Job Title, Department Name, Address/Mail Stop, and AIAA Member Grade (if any) for second author.}}
\affiliation{Business or Academic Affiliation 1, City, State, Zip Code}
\author{Third C. Author\footnote{Insert Job Title, Department Name, Address/Mail Stop, and AIAA Member Grade (if any) for third author.}}
\affiliation{Business or Academic Affiliation 2, City, Province, Zip Code, Country}
\author{Fourth D. Author\footnote{Insert Job Title, Department Name, Address/Mail Stop, and AIAA Member Grade (if any) for fourth author (etc.).}}
\affiliation{Business or Academic Affiliation 2, City, State, Zip Code}

\begin{abstract}

\end{abstract}

\maketitle

\section*{Nomenclature}
(Nomenclature entries should have the units identified)\\
\noindent\begin{tabular}{@{}lcl@{}}
\textit{A}  &=& amplitude of oscillation \\
\textit{a   }&=&    cylinder diameter \\
\textit{C}$_{p}$&=& pressure coefficient \\
\textit{Cx} &=& force coefficient in the \textit{x} direction \\
\textit{Cy} &=& force coefficient in the \textit{y} direction \\
c   &=& chord \\
d\textit{t} &=& time step \\
\textit{Fx} &=& \textit{X} component of the resultant pressure force acting on the vehicle \\
\textit{Fy} &=& \textit{Y} component of the resultant pressure force acting on the vehicle \\
\textit{f, g}   &=& generic functions \\
\textit{h}  &=& height \\
\textit{i}  &=& time index during navigation \\
\textit{j}  &=& waypoint index \\
\textit{K}  &=& trailing-edge (TE) nondimensional angular deflection rate
\end{tabular} \\

\section{Introduction}
-effecient small satellite launches are part of the future of spaceflight 

- Scramjets are part of an efficient small satellite solution

- SPARTAN is being developed at UQ

- the first stage of spartan must perform a trajectory manoeuvre that is different to most rocket flight
what should I cite here?

-cover some of the existing second stage results


\section{Combined Vehicle Analysis In CART3D}

\section{First Stage Optimised Trajectory}

- for 55kPa and 45kpa start points as well?


\section{Comparison With Two Stage Rocket}
compare the rocket stages with-and without- the scramjet

\end{document}
